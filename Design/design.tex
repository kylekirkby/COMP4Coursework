\chapter{Design}

\section{Overall System Design}

\subsection{Short description of the main parts of the system}

\begin{flushleft}
\textbf{Main Parts of the System}
 \\ \par These are the main parts of the proposed system.
	\begin{itemize}
			\item Proposed System User Interface
			\item Adding a New Job
			\item Adding a New Client
			\item Adding a New Material
			\item Sorting and Searching Clients
			\item Removing Clients or Jobs
			\item Calculating Costs For Each Job
			\item Generating Reports
			\item Invoice Output for Client
			\item Appointment Output for Client
	\end{itemize}

\end{flushleft}


\textbf{Proposed System User Interface}
	\begin{itemize}
		\item Once onto the proposed system, the plasterer will be able to see various buttons in the action bar at the top of the program; these include Add
Jobs, Clients, Materials.
		\item Pressing the Jobs Button will then take them to a different user interface which will then display a series of other options that are applicable under the Jobs section. These include Add Job, Delete Job, Search Jobs, Edit Job. 

	\end{itemize}
\textbf{Adding a New Job}
	\begin{itemize}
		\item The plasterer will click on a + new job button on the main window and will then be shown a different layout allowing the plasterer to add the details of a new job to the database.
		\item Whilst entering the job details various validation techniques will be implemented on the data entered into the fields. For example, there will be a post code regular expression validation to make sure that the post code entered is one of a correct and valid format.
		\item Once the form has been validated and submitted the plasterer will be show a success (or failure) message to let them know that it was added ok (or not).
	\end{itemize}

\textbf{Adding a New Client}

	\begin{itemize}
		\item The user will be able to click on a + new client button that will be situated on the main window and once clicked on, the user will be shown a new layout allowing the user to enter a new client and add all of their details to the application database.
		\item Whilst entering the new client info the data that is entered will be validated and the line edit will change colour depending on whether that data entered is valid or not. For the client details there will also be a regular expression validator attached to various fields including the client phone number, client email and client post code.
		\item Once the form data has been validated and is ok the data will be committed to the database and the user will be shown a message telling them whether it added the new client to the database successfully or not. They will then be returned to the main layout.
	\end{itemize}

\textbf{Adding a New Material}

	\begin{itemize}
		\item The user will be able to add multiple materials to the database to use when calculating the cost of a job.
		\item The user will be able to click a + new material button on the main window and then will be shown a new layout allowing them to enter the details and cost etc of the new material. The data entered will also be validated.
		\item Once the new material is added a success or failure message will be displayed to the user and then they will be returned to the main layout.
	\end{itemize}

\textbf{Sorting and Searching Clients}

	\begin{itemize}
		\item The user will be able to press a "Search Clients" button on the main window and then they will be taken to a new layout showing them a table with a list of the clients and their details. 
		\item Below the table widget their  will be a search field that allows the user to search for specific clients quickly and efficiently.
		\item The results will be updated on the text Changed event.
		\item The user will be able to sort the clients not only by search but other attributes such as town/city. When the user clicks the sort by town/city push button the results will update and be sorted by town.
	\end{itemize}

\textbf{Removing Clients and Jobs}

	\begin{itemize}
		\item Removing Clients will be a feature available in the clients section of the application. The user will be able to manage the clients in a "Manage Clients" area of the application.
		\item Their will be a table widget showing a list of all the clients and when a client record is clicked various options will become available to the user such as delete or edit. 
	\end{itemize}

\textbf{Calculating Costs for each Job}

	\begin{itemize}
		\item The calculation of the job will occur once the user clicks generate invoice for a specific job.
		\item This "Generate Invoice" button will be situated within the jobs section of the application (which will be available by clicking the "Jobs" button in the main layout).
		\item By clicking "Generate Invoice" the algoritm will collaborate all the available information regarding that specific job. See the algorithm section for more detail on this algorithm.
		\item In addition to this the user (plasterer) will be able to override the resulting calculations if need be before the final details are sent to the client.
	\end{itemize}

\textbf{Generate Reports}

	\begin{itemize}
		\item The reports section will be available by clicking a "Generate Report" button that will in the Jobs section of the application. 
		\item When the generate report button has been clicked a new layout will be displayed showing a table detailing the amount of money earned in a user specified period.
		\item The time period to generate a report for will be able to be edited through a form below the report table. The form will ask where/when/who to generate a report for.
		\item Reports will be able to be generated for different plasterers and different clients; or all clients and all plasterers etc.
	\end{itemize}

\textbf{Invoice Output for Client}

	\begin{itemize}
		\item An invoice for a job will be able to be generated by clicking a "Generate Invoice" button on the specific job. 
		\item Once clicked, the invoice will be displayed on screen in a preview box and then when the user is happy with it they can click "Print Invoice" to print the invoice or "Email Invoice" to email the invoice to the client through the email stored for the client within the database.
	\end{itemize}


\textbf{Appointment Output for Client}

	\begin{itemize}
		\item The appointments will be made by clicking the "Setup Appointment" button on the specific job an appointment is needed for.
		\item When clicked, a new layout will display allowing the user to select an appointment date and time for the client.
		\item Once validated and inserted into the database, the client will be emailed a copy of the appointment details.
	\end{itemize}


\pagebreak
\subsection{System flowcharts showing an overview of the complete system}
\textbf{Flow Chart Key}
\begin{flushleft}
This is the key for the following flow charts.
\end{flushleft}
\begin{figure}[H]
    \includegraphics[scale=0.4]{./Design/images/key.pdf}
    \caption{This is the flow chart key.} \label{fig:FlowChartKey}
\end{figure}


\pagebreak
\textbf{Main Menu Flow Chart}
\begin{flushleft}
This flow chart shows the options at the main menu in the application.
\end{flushleft}
\begin{figure}[H]
    \includegraphics[scale=0.4]{./Design/images/FlowChartMainMenu.pdf}
    \caption{This is the flow chart showing the Main Menu Selection.} \label{fig:FlowChartMainMenu}
\end{figure}


\pagebreak
\underline{\textbf{Client Flow Charts}}
\begin{flushleft}
The flow charts below show the options which can be selected for clients in the application; this includes adding a new client, searching clients and editing clients.
\end{flushleft}
\textbf{Client Menu Flow Chart}
\begin{flushleft}
This is the Client Menu flow chart which shows the options to select in the application regarding clients; this includes adding a new client, searching clients and editing clients.
\end{flushleft}
\begin{figure}[H]
\includegraphics[scale=0.3]{./Design/images/FlowChartClientMenu.pdf}
    \caption{This is the flow chart showing the Client Menu Selection.} \label{fig:FlowChartClientMenu}
\end{figure}

\pagebreak
\textbf{New Client Flow Chart}
\begin{flushleft}
Below is a flow chart to show what happens when you add a new client in the proposed system.
\end{flushleft}
\begin{figure}[H]
    \includegraphics[scale=0.5]{./Design/images/FlowChartNewClient.pdf}
    \caption{This is the flow chart showing the addition of a new client to the proposed system.} 
\label{fig:FlowChartNewClient}
\end{figure}

\pagebreak
\textbf{Current Clients Flow Chart}
\begin{flushleft}
Below is the current clients flow chart which shows the flow of control in the proposed system for this section.
\end{flushleft}
\begin{figure}[H]
    \includegraphics[scale=0.5]{./Design/images/FlowChartCurrentClients.pdf}
    \caption{This is the Current Clients Flow Chart.} 
\label{fig:FlowChartCurrentClients}
\end{figure}

\pagebreak
\textbf{Search Clients Flow Chart}
\begin{flushleft}
Below you can see how you will be able to search the clients in the proposed system. You will be able to search via Location or Search Term etc.
\end{flushleft}
\begin{figure}[H]
\includegraphics[scale=0.5]{./Design/images/FlowChartSearchClients.pdf}
    \caption{This is the Search Clients Flow Chart.} 
\label{fig:FlowChartSearchClients}
\end{figure}

\pagebreak
\textbf{Manage Clients Flow Chart}
\begin{flushleft}
Below is the manage clients flow chart which shows what happens when you click manage clients in the proposed system.
\end{flushleft}

\begin{figure}[H]
\includegraphics[scale=0.5]{./Design/images/FlowChartManageClients.pdf}
    \caption{This is the Manage Clients Flow Chart.} 
\label{fig:FlowChartSearchClients}
\end{figure}




\pagebreak
\underline{\textbf{Plasterer Flow Charts}}
\par
\begin{flushleft}
The flow charts below are from the plasterer section of the application. They show the options that can be selected for plasterers such as searching,editing and creating new plasterers.
\end{flushleft}

\textbf{Plasterer Menu Flow Chart}
\begin{flushleft}
This is the plasterer menu flow chart showing what users can do in the plasterers section of the application.
\end{flushleft}

\begin{figure}[H]
\includegraphics[scale=0.4]{./Design/images/FlowChartPlastererMenu.pdf}
    \caption{This is the Plasterer Menu Flow Chart.} 
\label{fig:FlowChartPlastererMenu}
\end{figure}


\pagebreak
\textbf{Adding a New Plasterer}
\begin{flushleft}
This flow chart shows what happens in the application when you add a new plasterer to the system.
\end{flushleft}

\begin{figure}[H]
\includegraphics[scale=0.5]{./Design/images/FlowChartNewPlasterer.pdf}
    \caption{This is the New Plasterer Flow Chart.} 
\label{fig:FlowChartNewPlasterer}
\end{figure}


\pagebreak
\textbf{Current Plasterers}
\begin{flushleft}
This flow chart shows the current plasterers options from within the application.
\end{flushleft}

\begin{figure}[H]
\includegraphics[scale=0.5]{./Design/images/FlowChartCurrentPlasterers.pdf}
    \caption{This is the Current Plasterers Flow Chart.} 
\label{fig:FlowChartCurrentPlasterers}
\end{figure}


\pagebreak
\textbf{Manage Plasterers}
\begin{flushleft}
This flow chart shows the manage plasterers window from the proposed application.
\end{flushleft}
\begin{figure}[H]
\includegraphics[scale=0.5]{./Design/images/FlowChartManagePlasterers.pdf}
    \caption{This is the Manage Plasterers Flow Chart.} 
\label{fig:FlowChartManagePlasterers}
\end{figure}

\pagebreak
\textbf{Search Plasterers}
\begin{flushleft}
This flow chart shows the search plasterers feature of the proposed application. It lets the user search the plasterers for specific records.
\end{flushleft}
\begin{figure}[H]
\includegraphics[scale=0.5]{./Design/images/FlowChartSearchPlasterers.pdf}
    \caption{This is the Search Plasterers Flow Chart.} 
\label{fig:FlowChartSearchPlasterers}
\end{figure}



\pagebreak
\underline{\textbf{Jobs Flow Charts}}
\begin{flushleft}
The flow charts below are from the jobs section of the application.
\end{flushleft}
\textbf{Job Menu Flow Chart}
\begin{flushleft}
This flow chart shows the selection choice at the job menu section of the application.
\end{flushleft}
\begin{figure}[H]
\includegraphics[scale=0.5]{./Design/images/FlowChartJobMenu.pdf}
    \caption{This is the Job Menu Flow Chart.} 
\label{fig:FlowChartJobMenu}
\end{figure}

\pagebreak
\textbf{New Job Flow Chart}
\begin{flushleft}
This flow chart shows what happens in the application when the users adds a new job.
\end{flushleft}
\begin{figure}[H]
\includegraphics[scale=0.5]{./Design/images/FlowChartNewJob.pdf}
    \caption{This is the New Job Flow Chart.} 
\label{fig:FlowChartNewJob}
\end{figure}


\pagebreak
\textbf{Manage Jobs Flow Chart}
\begin{flushleft}
This shows the manage jobs section of the application.
\end{flushleft}
\begin{figure}[H]
\includegraphics[scale=0.5]{./Design/images/FlowChartManageJobs.pdf}
    \caption{This is the Manage Jobs Flow Chart.} 
\label{fig:FlowChartManageJobs}
\end{figure}


\pagebreak
\section{User Interface Designs}
\textbf{Main Menu To Clients Menu UI}
\begin{flushleft}
This shows what the user interface will look and behave like when the clients button is clicked.
\end{flushleft}
\begin{figure}[H]
\includegraphics[scale=0.5]{./Design/images/UI-ClientsMenu.pdf}
    \caption{This is the Clients Menu User Interface.} 
\label{fig:FlowChartClientsMenu}
\end{figure}

\pagebreak
\textbf{Clients Menu to New Client UI}
\begin{flushleft}
This User Interface diagram shows what happens in the proposed system when the users adds a new client.
\end{flushleft}
\begin{figure}[H]
\includegraphics[scale=0.5]{./Design/images/UI-NewClient.pdf}
    \caption{This is the New Client User Interface.} 
\label{fig:FlowChartNewClient}
\end{figure}

\pagebreak
\textbf{Clients Menu to Current Clients UI}
\begin{flushleft}
This User Interface diagram shows what happens in the proposed system when the clicks current clients in the client menu.
\end{flushleft}
\begin{figure}[H]
\includegraphics[scale=0.5]{./Design/images/UI-CurrentClients.pdf}
    \caption{This is the Current Clients User Interface.} 
\label{fig:FlowChartNewClient}
\end{figure}

\pagebreak
\textbf{Current Clients Menu to Search Clients UI}
\begin{flushleft}
This User Interface diagram shows what happens in the proposed system when the clicks search clients in the current clients menu.
\end{flushleft}
\begin{figure}[H]
\includegraphics[scale=0.5]{./Design/images/UI-SearchClients.pdf}
    \caption{This is the Search Clients UI} 
\label{fig:FlowChartNewClient}
\end{figure}


\pagebreak
\textbf{Main Menu to Jobs Menu UI}
\begin{flushleft}
This User Interface diagram shows what happens in the proposed system when the clicks jobs in the main menu.
\end{flushleft}
\begin{figure}[H]
\includegraphics[scale=0.5]{./Design/images/JobsMenu.pdf}
    \caption{This is the Jobs Menu UI} 
\label{fig:FlowChartJobsMenu}
\end{figure}


\pagebreak
\textbf{Jobs Menu to Add Job UI}
\begin{flushleft}
This User Interface diagram shows what happens in the proposed system when the clicks add job in the jobs menu.
\end{flushleft}
\begin{figure}[H]
\includegraphics[scale=0.5]{./Design/images/NewJob.pdf}
    \caption{This is the Add Job UI} 
\label{fig:FlowChartAddJob}
\end{figure}


\pagebreak
\textbf{Manage Jobs UI Design}
\begin{flushleft}
This User Interface diagram shows what happens in the proposed system when the clicks manage jobs in the jobs menu.
\end{flushleft}
\begin{figure}[H]
\includegraphics[scale=0.5]{./Design/images/ManageJobs.pdf}
    \caption{This is the Manage Jobs UI} 
\label{fig:FlowChartManageJobs}
\end{figure}



\pagebreak
\textbf{Main Menu to Plasterers Menu UI}
\begin{flushleft}
This User Interface diagram shows what happens in the proposed system when the clicks plasterers in the main menu.
\end{flushleft}
\begin{figure}[H]
\includegraphics[scale=0.5]{./Design/images/PlasterersMenu.pdf}
    \caption{This is the Plasterers Menu UI} 
\label{fig:FlowChartPlasterersMenu}
\end{figure}


\pagebreak
\textbf{Add Plasterer UI}
\begin{flushleft}
This User Interface diagram shows what happens in the proposed system when the clicks add plasterer in the plasterers menu.
\end{flushleft}
\begin{figure}[H]
\includegraphics[scale=0.5]{./Design/images/NewPlasterer.pdf}
    \caption{This is the New Plasterer UI} 
\label{fig:FlowChartNewPlasterer}
\end{figure}

\pagebreak
\textbf{Manage Plasterer UI Design}
\begin{flushleft}
This User Interface diagram shows what happens in the proposed system when the clicks manage plasterers in the plasterers menu.
\end{flushleft}
\begin{figure}[H]
\includegraphics[scale=0.5]{./Design/images/ManagePlasterers.pdf}
    \caption{This is the Manage Plasterers UI} 
\label{fig:FlowChartManagePlasterers}
\end{figure}

\pagebreak
\section{Hardware Specification}

\begin{flushleft}
	Dan currently has a Toshiba Laptop with the following hardware components:
	
	\begin{itemize}
		\item 6GB DDR3 RAM
		\item Intel 2.0ghz Dual Core Processor
		\item High Resolution 1920 x 1080 Display
		\item 1 TB HDD
		\item Intel Onboard Integrated Graphics
	\end{itemize}

	The specification of Dan's laptop is more than powerful enough and has enough RAM to run the proposed python application alongside multiple other programs that he regularly uses (such as a Web Browser and Media Player). The hard drive has enough storage space to install the application (which will only be around 10MB) so storage is not a problem. The only possible complication when it comes to hardware may be the screen resolution as the application will have to keep within the resolution and be optimized for this screen size and many more to keep the software versatile.
\end{flushleft}

\subsection{Software}

\begin{flushleft}
	Dan is running the Windows 8 operating system on his laptop which does not have Python 3 installed. The proposed system will need to be able to run on this operating system therefore it is a software constraint that will need to be addressed. This could be achieved by building an installer for the python application so it can be installed on Windows 8 as pythons libraries are included.

\end{flushleft}


\pagebreak
\section{Program Structure}
\subsection{Top-down design structure charts}
\textbf{Top Down System Structure Chart}
\begin{flushleft}
The structure chart below shows the proposed systems structure from a top down approach.
\end{flushleft}

\begin{figure}[H]
\includegraphics[scale=0.4]{./Design/images/TopDownChart.pdf}
    \caption{Top Down Chart} 
\label{fig:FlowChartNewClient}
\end{figure}






\pagebreak
\subsection{Algorithms in pseudo-code for each data transformation process}

\textbf{Algorithm for Generating an Invoice from Job Details}
\begin{flushleft}
This algorithm shows how the data transforms from one medium to another in regards to generating an invoice for the client. The algoritm takes the data stored for the job from the JobMaterials Table, the plasterers daily rate and number of days worked on the job then uses this information to calculate the total cost of the job to be displayed on the invoice.
\end{flushleft}
\begin{algorithm}[H]
\label{fig:generating_an_invoice_example}
\caption{Genreating an Invoice Algorithm}
\begin{algorithmic}[1]
\Function{$GenerateInvoice$}{$JobID$}
	\SET{$Total$}{$0$}
	\State
	\SET{$DailyRate$}{GetDailyRate(JobID)}
	\SET{$DaysWorked$}{GetDaysWorked(JobID)}
	\SET{$JobMaterials$}{GetJobMaterialsList(JobID)}
	\State
	\For{$count$}{$0$}{$len(JobMaterials)$}
		\SET{$Price$}{JobMaterials[count]["Price"]}
		\SET{$Quantity$}{JobMaterials[count]["Quantity"]}
		\State
		\SET{$MaterialCost$}{Price * Quantity}
		\State
		\SET{$Total$}{Total + MaterialCost}
	\EndFor
	\State
	\SET{$PlastererWage$}{DailyRate * DaysWorked}
	\SET{$Total$}{Total + PlasterersWage}
	\State
	\Return{$Total$}
\EndFunction
\end{algorithmic}
\end{algorithm}

\pagebreak
\textbf{Algorithm for calculating pay for plasterer in time period.}
\begin{flushleft}
This algorithm shows how the proposed system will generate the necessary data for displaying graphs about the amount of money a plasterer has earned within a given time period. The algorithm is a function which takes a start and end date along with a plasterer id. It then proceeds to collect the pay details of jobs that the plasterer has done.
\end{flushleft}
\begin{algorithm}[H]
\label{fig:calculating_plasterer_pay}
\caption{Calculating Plasterer Pay}
\begin{algorithmic}[1]
\Function{$CalculatePay$}{$PlastererID$,$StartDate$,$EndDate$}

	\SET{$Total$}{$0$}
	\State
	\SET{$AllJobsList$}{GetAllJobs(PlastererID)}
	\State
	\For{$count$}{$0$}{$len(AllJobsList)$}
		\If{$AllJobsList[count]["JobDate"]$ "LESS THAN OR EQUAL TO" $EndDate$ AND $AllJobsList[count]["JobDate"]$ "GREATER THAN OR EQUAL TO" $StartDate$}
			\State
			\SET{$Price$}{AllJobsList[count]["InvoiceTotal"]}
			\State
			\SET{$Total$}{Total + Price}
		\EndIf
	\EndFor
	\State
	\Return{$Total$}
\EndFunction
\end{algorithmic}
\end{algorithm}

\pagebreak
\textbf{Algorithm for printing an invoice.}
\begin{flushleft}
This algorithm shows how the proposed system will gather the invoice details and print the invoice by the job id specified.
\end{flushleft}
\begin{algorithm}[H]
\label{fig:printing_an_invoice}
\caption{Printing an Invoice}
\begin{algorithmic}[1]
\Function{$PrintInvoice$}{$JobID$}
	\State
	\SET{$JobInvoices$}{GetJobInvoice(JobId)}
	\State
	\If{$len(JobInvoices) == 0$}
		\SET{$Invoice$}{$GenerateInvoice(JobID)$}
	\Else
		\SET{$Invoice$}{$JobInvoices$}
	\State
	\EndIf
	\State
	\SET{$Printed$}{$PrintWithQPrinter(Invoice)$}
	\State
	\Return{$Printed$}
\EndFunction
\end{algorithmic}
\end{algorithm}


\pagebreak
\subsection{Object Diagrams}
\begin{flushleft}
This Diagram shows the relationship between the objects in the system.
\end{flushleft}
\begin{figure}[H]
\includegraphics[scale=0.5]{./Design/images/ObjectDiagram.pdf}
    \caption{Object Relationships} 
\label{fig:ObjectDiagram}
\end{figure}



\pagebreak
\subsection{Class Definitions}

\begin{flushleft}
\textbf{Key:}
\begin{tabular}{|p{4cm}|}
	\hline
	Label \\ \hline
	Attributes \\ \hline	
	Behavious \\ \hline
\end{tabular}


\end{flushleft}

\begin{tabular}{|p{4cm}|}
	\hline
	\textbf{Member} \\ \hline
	MemberID \\
	MemberTitle \\
	MembeFirstName \\
	MemberSurname \\
	MemberAddrLine1 \\
	MemberAddrLine2 \\
	MemberAddrLine3 \\
	MemberAddrLine4 \\
	MemberEmail \\
	MemberPhoneNumber \\ \hline	
		AddMemberTitle \\
		AddMemberFirstName \\
		AddMemberSurname \\
		AddMemberAddrLine1 \\
		AddMemberAddrLine2 \\
		AddMemberAddrLine3 \\
		AddMemberAddrLine4 \\
		AddMemberEmail \\
		AddMemberPhoneNumber \\ 
		EditMemberTitle \\
		EditMemberFirstName \\
		EditMemberSurname \\
		EditMemberAddrLine1 \\
		EditMemberAddrLine2 \\
		EditMemberAddrLine3 \\
		EditMemberAddrLine4 \\
		EditMemberEmail \\
		EditMemberPhoneNumber \\ \hline
\end{tabular}

\begin{tabular}{|p{5cm}|}
	\hline
	\textbf{Client:} extends \textbf{Member} \\ \hline
		 \\ \hline
		 \\ \hline
\end{tabular}


\begin{tabular}{|p{5cm}|}
	\hline
	\textbf{Plasterer:} extends \textbf{Member} \\ \hline
		PlastererDailyRate \\ \hline
		AddPlastererDailyRate \\
		EditPlastererDailyRate \\ \hline
\end{tabular}

\begin{tabular}{|p{5cm}|}
	\hline
	\textbf{Job:} \\ \hline
		JobID \\
		ClientID \\
		PlastererID \\
		JobDescription \\
		JobAddrLine1 \\
		JobAddrLine2 \\
		JobAddrLine3 \\
		JobAddrLine4 \\
		JobDaysWorked \\
		JobComplete \\
		JobPaid \\ 
		InvoiceID \\ \hline
		AddJobDescription\\
		AddJobAddrLine1 \\
		AddJobAddrLine2 \\
		AddJobAddrLine3 \\
		AddJobAddrLine4 \\
		AddJobDaysWorked \\
		EditJobComplete \\
		EditJobPaid \\
		EditJobDescription\\
		EditJobAddrLine1 \\
		EditJobAddrLine2 \\
		EditJobAddrLine3 \\
		EditJobAddrLine4 \\
		EditJobDaysWorked \\ \hline
\end{tabular}

\begin{tabular}{|p{5cm}|}
	\hline
	\textbf{Material:} \\ \hline
		MaterialID \\
		MaterialName \\
		MaterialPrice \\ \hline
		AddMaterialName\\
		AddMaterialPrice \\ \hline	
		EditMaterialName \\
		EditMaterialPrice \\ \hline
\end{tabular}

\begin{tabular}{|p{5cm}|}
	\hline
	\textbf{Document:} \\ \hline
		DocumentID \\
		DocumentDate \\
		DocumentTime \\
		DocumentText \\ \hline
		AddDocumentDate \\
		AddDocumentTime \\
		AddDocumentText \\
		EditDocumentDate \\
		EditDocumentTime \\
		EditDocumentText \\ \hline
		
\end{tabular}


\begin{tabular}{|p{5cm}|}
	\hline
	\textbf{Invoice:} extends \textbf{Document} \\ \hline
		ClientID \\
		JobID \\
		PlastererID\\
		InvoiceAmountPreTax \\
		InvoiceAmountAfterTax \\
		InvoiceReceived \\ \hline
		AddInvoiceAmountPreTax \\ 
		AddInvoiceAmountAfterTax \\
		AddInvoiceReceived \\
		EditInvoiceAmountPreTax \\ 
		EditInvoiceAmountAfterTax \\
		EditInvoiceReceived \\ \hline
		
\end{tabular}

\begin{tabular}{|p{5cm}|}
	\hline
	\textbf{Appointment:} extends \textbf{Document} \\ \hline
		ClientID \\
		PlastererID \\
		AppointmentAddrLine1 \\
		AppointmentAddrLine2 \\
		AppointmentAddrLine3 \\
		AppointmentAddrLine4 \\ \hline
		AddAppointmentAddrLine1 \\
		AddAppointmentAddrLine2 \\
		AddAppointmentAddrLine3 \\
		AddAppointmentAddrLine4 \\
		EditAppointmentAddrLine1 \\
		EditAppointmentAddrLine2 \\
		EditAppointmentAddrLine3 \\
		EditAppointmentAddrLine4 \\ \hline

\end{tabular}

\section{Prototyping}

\textbf{PyQt4 Printing Prototyping}
\begin{flushleft}
I researched and developed a small printing prototype using PyQt4 and Python3. I found information on printing with PyQt4 in the book \emph{Rapid GUI Programming with Python and Qt} By \emph{Mark Summerfield}. The book described how you can print using a QTextDocument, QPrinter and HTML formatting. Below is the code for this prototype.
\end{flushleft}
\begin{python}
from PyQt4.QtGui import *
from PyQt4.QtCore import *
import sys


class MainWindow(QMainWindow):
    """ This is a test for the printing function """

    def __init__(self):
        super().__init__()


        self.setWindowTitle("Printing Test")
        self.printer = QPrinter()
        self.printer.setPageSize(QPrinter.Letter)
        self.mainLayout()

        
    def mainLayout(self):

        self.layout = QHBoxLayout()
        self.printButton = QPushButton("Print")

        self.layout.addWidget(self.printButton)

        self.mainWidget = QWidget()
        self.mainWidget.setLayout(self.layout)

        self.setCentralWidget(self.mainWidget)

        self.printButton.clicked.connect(self.printViaHtml)

    def getCurrentDate(self,dateFormat):
        date = QDate.currentDate().toString(dateFormat)
        return date

    def statementHtml(self):

        companyName = "DnA Plastering"
        date = self.getCurrentDate("dd.MM.yyyy")

        jobItems = [["25KG Plaster","30.00","5","150.00"],["Angle Beading 3M","7.00","5","35.00"]]

        invoiceId = "4564"
        amountDue = 0.0
        for each in jobItems:
            price = float(each[3])
            amountDue += price
        amountAfterTax = amountDue * 1.2
        address = ["15 The Glebe","Haverhill","Suffolk","CB9 0DL"]
        
            
            
        
        html = u""
        html += ("<h1 align='center'>{0}</h1>"
                 "<table width='30%' align='left' cellpadding='10px'>").format(companyName)

        for each in address:
            html += ("<tr><td>{0}</td></tr>").format(each)


        html += ("</table>"
                 "<table width='30%' border='1' align='right' cellpadding='10px'>"
                 "<tr><td><b>Invoice</b> #</td><td>{0}</td></tr>"
                 "<tr><td><b>Date</b></td><td>{1}</td></tr>"
                 "<tr><td><b>Amount Due</b></td><td>{2}</td></tr>"
                 "</table>"
                 "<br><hr/>"
                 "<table width='100%' border='1' cellpadding='10px'>"
                 "<tr><td><b>Material/Item</b></td><td><b>Unit Cost</b></td><td><b>Quantity</b></td><td><b>Price</b></td></tr>").format(invoiceId,date,amountDue)

        for item in jobItems:
            html += ("<tr><td>{0}</td><td>{1}</td><td>{2}</td><td>{3}</td></tr>").format(item[0],item[1],item[2],item[3])

        html += ("<tr><td></td><td></td><td><b>Sub Total</b></td><td>{0}</td></tr>"
                 "<tr><td></td><td></td><td><b>Total (20% VAT added)</b></td><td>{1}</td></tr>").format(amountDue,amountAfterTax)

        html += ("</table>"
                 "<br>"
                 "<hr/>"
                 "<p align='center'>{0}</p>").format(companyName)


        return html

        

    def printViaHtml(self):

        html = self.statementHtml()

        dialog = QPrintDialog(self.printer, self)
        if dialog.exec_():
            document = QTextDocument()
            document.setHtml(html)
            document.print_(self.printer)
        else:
            print("The print process has failed!")

        print(html)
        


if __name__ == "__main__":

    app = QApplication(sys.argv)
    window = MainWindow()
    window.show()
    window.raise_()
    app.exec_()


\end{python}

\pagebreak
\section{Definition of Data Requirements}

\subsection{Identification of all data input items}

\begin{flushleft}
Below are all of the data input items in the proposed system. The values are entered by the proposed systems end user through 
the applications user interface.

\end{flushleft}


\begin{flushleft}

\begin{longtable}{|p{3.5cm}|p{5cm}|} \hline
\textbf{Data} & \textbf{Description} \\ \hline
ClientTitle & The title of the client (Mr,Mrs, etc). \\ \hline
ClientFirstName & The clients first name \\ \hline
ClientSurname & The clients surname. \\ \hline
ClientAddrLine1 & The clients street. \\ \hline
ClientAddrLine2 & The clients town/city. \\ \hline
ClientAddrLine3 & The clients county. \\ \hline
ClientAddrLine4 & The clients post code. \\ \hline
ClientEmail & The clients email address. \\ \hline
ClientPhoneNumber & The clients phone number. \\ \hline \hline

PlastererTitle & The title of the plasterer (Mr,Mrs etc) \\ \hline
PlastererFirstName & The surname of the plasterer. \\ \hline
PlastererSurname & The surname of the plasterer. \\ \hline
PlastererAddrLine1 & The plasterers street. \\ \hline
PlastererAddrLine2 & The plasterers town/city. \\ \hline
PlastererAddrLine3 & The plasterers county. \\ \hline
PlastererAddrLine4 & The plasterers post code. \\ \hline
PlastererEmail & The plasterers email. \\ \hline
PlastererPhoneNumber & The plasterers phone number. \\ \hline
PlastererDailyRate & The plasterers daily working rate. \\ \hline \hline


JobDescription & A brief description of the job. \\ \hline
JobAddrLine1 & The Jobs street. \\ \hline
JobAddrLine2 & The Jobs town/city. \\ \hline
JobAddrLine3 & The Jobs county. \\ \hline
JobAddrLine4 & The Jobs post code. \\ \hline
JobDaysWorked & The number of days the plasterer worked on this job. \\ \hline
JobComplete & A boolean value to show whether the job is complete or not. \\ \hline \hline

MaterialName & The Name of the material \\ \hline
MaterialPrice & The cost of the material \\ \hline
JobMaterialsQuantity & The number of this material used \\ \hline \hline

InvoiceDate & The date the invoice is sent \\ \hline
InvoiceAmountPreTax & The invoice amount before VAT is added \\ \hline
InvoiceAmountAfterTax & The invoice amount after VAT is added \\ \hline
InvoiceReceived & A boolean to show whether the invoice has been received \\ \hline
InvoiceText & Text Description to go on the invoice \\ \hline 
InvoicePaid & A boolean to show whether the job has been paid for \\ \hline \hline

AppointmentTime & The time of the appointment \\ \hline
AppointmentDate & The date of the appointment \\ \hline 

\end{longtable}

\end{flushleft}


\subsection{Identification of all data output items}
\begin{flushleft}
The data outputs in the proposed system these are listed below.
\end{flushleft}
\begin{itemize}
\item Invoices (Print or Email)
\item Appointments (Email)
\item Graphs
\end{itemize}

\subsection{Explanation of how data output items are generated}

\textbf{Apppointment}
\begin{flushleft}
The appointment ouput in the proposed system is generated by taking the AppointmentDate and AppointmentTime that the user enters and then forming a text document that can be sent to the clients stored email address.
\end{flushleft}

\textbf{Invoices}
\begin{flushleft}
The invoices collect the records for a job from the JobMaterials table and then calculates accordingly the cost of the materials involved. Once this has been done then the number of days worked is multiplied by the plasterers daily rate. All of this is put into HTML format and will then be ready to print or send to the clients email.
\end{flushleft}

\textbf{Graphs}
\begin{flushleft}
The graphs in the proposed system will allow the end user the ability to track their earnings over the past few months from jobs that they have done. This may be done for individual plasterers to see what they have paid in tax and earned during a user defined time period.
\end{flushleft}


\pagebreak
\subsection{Data Dictionary}


\begin{flushleft}
\begin{longtable}{|p{3.6cm}|p{1.5cm}|p{2cm}|p{2cm}|p{2cm}|p{2cm}|}
\hline
\textbf{Name} & \textbf{Data Type} & \textbf{Length} & \textbf{Validation} & \textbf{Example Data} & \textbf{Comment} \\ \hline



ClientID & Integer & 1 - 1000 & Range & 1, 200, 45 & Primary Key \\ \hline
ClientTitle & String & 2 - 10 Chars & Length & Mr,Mrs,Sir &  \\ \hline 
ClientFirstName & String & 3 - 25 Chars & Length & Dan, kyle &  \\ \hline 
ClientSurname & String & 3 - 35 Chars & Length & Austin, Kirkby &  \\ \hline 
ClientAddrLine1 & String & 5 - 30 Chars & Length & 15 Crowley Road &  Street name \\ \hline 
ClientAddrLine2 & String & 5 - 30 Chars & Length & Haverhill & Town / City \\ \hline 
ClientAddrLine3 & String & 5 - 30 Chars & Length & Suffolk & County  \\ \hline 
ClientAddrLine4 & String & 6 - 7 Chars & Length and Format Check & CB9 0DJ & Post Code\\ \hline 
ClientEmail & String & 7 - 50 Chars & Length and format check & dan@ gmail.com &  \\ \hline 
ClientPhoneNumber & String & 11 Chars & Length & 07809 726 812 &  \\ \hline \hline



PlastererID & Integer & 1 - 50  & Range  & 1, 2, 45 & Primary Key \\ \hline
PlastererFirstName & String & 3 - 30 Chars & Length & Dan, Kyle & \\ \hline
PlastererSurname & String & 3 - 30 Chars & Length & Austin, Kirkby & \\ \hline
PlastererAddrLine1 & String & 5 - 30 Chars & Length & 15 Long Road & Street Name \\ \hline
PlastererAddrLine2 & String & 5 - 30 Chars & Length & Cambridge & Town / City \\ \hline
PlastererAddrLine3 & String & 5 - 30 Chars & Length & Essex & County \\ \hline
PlastererAddrLine4 & String & 6 - 7 Chars & Length and Format Check & CB2 5HD & Post Code \\ \hline
PlastererEmail & String & 7 - 50 Chars & Length and Format Check & dan@ gmail.com  &  \\ \hline
PlastererPhoneNumber & String & 11 Chars & Length & 07710 300 678 & \\ \hline
PlastererDailyRate & Currency & 40 - 250 & Range Check & 70, 150, 200 & \\ \hline \hline



JobID & Integer & 1 - 2000 & Range Check & 1, 3, 5 & Primary Key \\ \hline
JobDescription & Text & 10 - 1000 Chars & Range Check & 5m x 4m x 3m Living Room to be plastered. & Description of job. \\ \hline
JobAddrLine1 & String & 5 - 30 Chars & Length & 15 Long Road & Street Name \\ \hline
JobAddrLine2 & String & 5 - 30 Chars & Length & Cambridge & Town / City \\ \hline
JobAddrLine3 & String & 5 - 30 Chars & Length & Essex & County \\ \hline
JobAddrLine4 & String & 6 - 7 Chars & Length and Format Check & CB2 5HD & Post Code \\ \hline
JobDaysWorked & Integer & 0 - 30 & Range Check & 1, 7 14 &  \\ \hline
JobComplete & Boolean &  & Presence Check & TRUE, FALSE &  \\ \hline
JobPaid & Boolean &  & Presence Check & TRUE, FALSE &  \\ \hline \hline



MaterialID & Integer & 1 - 50 & Range Check & 3, 4, 29 & Primary Key \\ \hline
MaterialName & String & 3 - 50 & Length & Angle Beading &  \\ \hline  
MaterialPrice & Currency & 1 - 1000 & Range Check & 1, 3, 20 &  \\ \hline \hline


JobMaterialsID & Integer & 1 - 4000 & Range Check & 1, 300, 3563 & Primary Key \\ \hline
JobMaterialsQuantity & Integer & 1 - 500 & Range Check & 59, 245, 309 & \\ \hline \hline 


InvoiceID & Integer & 1 - 2000 & Range Check & 1, 2, 45 & Primary Key \\ \hline
InvoiceAmountPreTax & Currency & 50 - 7000 & Range Check & 140, 890, 1050 &  \\ \hline
InvoiceAmountAfterTax & Currency & 50 - 7000 & Range Check & 100, 800, 900 &  \\ \hline
InvoiceReceived & Boolean &  & Presence Check & TRUE, FALSE &  \\ \hline
InvoiceDate & DateTime &  & Presence Check & 14/12/2013 14:20 &  \\ \hline
InvoiceText & Text & 1 - 1000 & Length & 5 Days Worked &  \\ \hline \hline



AppointmentID & Integer & 1 - 2500 & Range Check & 1, 2, 45 & Primary Key \\ \hline
AppointmentDate & Date & & Presence Check  & 07/12/2013  & \\ \hline
AppointmentTime & Time &  & Presence Check  & 16:10 & \\ \hline
AppointmentAddrLine1 & String & 5 - 30 Chars & Length & 15 Long Road & Street Name \\ \hline
AppointmentAddrLine2 & String & 5 - 30 Chars & Length & Cambridge & Town / City \\ \hline
AppointmentAddrLine3 & String & 5 - 30 Chars & Length & Essex & County \\ \hline
AppointmentAddrLine4 & String & 6 - 7 Chars & Length and Format Check & CB2 5HD & Post Code \\ \hline

\end{longtable}
\end{flushleft}

\subsection{Identification of appropriate storage media}

\begin{flushleft}
In the proposed system the secondary storage required to run the application will only need to be basic as there is not too much data being stored. In the Analysis section the volumetrics of the proposed system suggested that the system would only require around 10MB. But with additional clients, jobs, materials, invoices, appointments etc the databse will increase in size and more storage would be required; Dan's computer has around 800GB free hard disk space so this is more than enough to run the proposed system for the forseeable future.

An external hard drive dan also has will suit as appropriate storage media for back ups of the data. This is because the external hard drive has 1TB of storage which is plenty for the amount of data the proposed system will produce. Using this External HDD, Dan will be able to make regular back ups of the data keeping the vital databases of client and job data secure.
\end{flushleft}

\pagebreak
\section{Database Design}

\subsection{Normalisation}

\subsubsection{ER Diagrams}

\begin{figure}[H]
    \includegraphics[scale=0.4]{./Design/images/ERDiagram.pdf}
    \caption{This is the entity relationship diagram for the sqlite3 database.} \label{fig:Entity_Relationship_Diagram}
\end{figure}


\pagebreak
\subsubsection{Entity Descriptions}

\begin{flushleft}

Below are the entity descriptions for the various entites in the proposed system. An \underline{underlined} attribute denotes a primary key and a \emph{emphasised}	attribute signifies a foreign key in the entity.

\end{flushleft}




\begin{flushleft}
	\textbf{Client}(\underline{ClientID}, ClientTitle, ClientFirstName, ClientSurname, ClientAddrLine1, ClientAddrLine2, ClientAddrLine3, ClientAddrLine4, ClientEmail, ClientPhoneNumber)
\end{flushleft}



\begin{flushleft}
	\textbf{Plasterer}(\underline{PlastererID}, PlastererFirstName, PlastererSurname, PlastererAddrLine1, PlastererAddrLine2, PlastererAddrLine3, PlastererAddrLine4, PlastererEmail,PlastererPhoneNumber, PlastererDailyRate)
\end{flushleft}



\begin{flushleft}
\textbf{Job}(\underline{JobID}, \emph{ClientID}, \emph{PlastererID}, JobDescription, JobAddrLine1, JobAddrLine2, JobAddrLine3, JobAddrLine4, JobDaysWorked,  JobComplete, \emph{InvoiceID})
\end{flushleft}


\begin{flushleft}
\textbf{Material}(\underline{MaterialID},MaterialName,MaterialPrice)
\end{flushleft}


\begin{flushleft}
\textbf{JobMaterials}(\underline{JobMaterialsID}, \emph{JobID}, \emph{MaterialsID}, JobMaterialsQuantity)
\end{flushleft}


\begin{flushleft}
\textbf{Invoice}(\underline{InvoiceID}, \emph{JobID}, InvoiceAmountPreTax, InvoiceAmountAfterTax, InvoiceReceived, InvoiceDate, InvoiceText, InvoicePaid)
\end{flushleft}



\begin{flushleft}
\textbf{Appointment}(\underline{AppointmentID}, \emph{JobID} AppointmentDate, AppointmentTime)
\end{flushleft}

\subsubsection{1NF to 3NF}
\begin{flushleft}
    \begin{longtable}{|p{12cm}|}
        \hline
			 \textbf{1NF} \\ \hline
         \textbf{Non-Repeating Attributes} \\ \hline
			PersonID (Primary Key) \\ 
         ClientTitle \\
			ClientFirstName \\
			ClientSurname \\
			ClientAddrLine1 \\
			ClientAddrLine2 \\
			ClientAddrLine3 \\
			ClientAddrLine4 \\
			ClientEmail \\
			ClientPhoneNumber \\
			PlastererTitle \\
			PlastererFirstName \\
			PlastererSurname \\
			PlastererAddrLine1 \\
			PlastererAddrLine2 \\
			PlastererAddrLine3 \\
			PlastererAddrLine4 \\
			PlastererEmail \\
			PlastererPhoneNumber \\
			PlastererDailyRate \\ \hline

			\textbf{Repeating Attributes} \\ \hline
			JobID (Primary Key) \\
			PersonID (Composite Key) \\
         JobDescription \\
			JobAddrLine1 \\
			JobAddrLine2 \\
			JobAddrLine3 \\
			JobAddrLine4 \\
			JobDaysWorked \\
			JobComplete \\
			MaterialName \\
			MaterialPrice \\
			JobMaterialsQuantity \\
			InvoiceAmountAfterTax \\
			InvoiceAmountPreTax \\
			InvoiceReceived \\
			InvoiceDate \\
			InvoiceText \\
			InvoicePaid \\
			AppointmentDate \\
			AppointmentTime \\
			AppointmentAddrLine1 \\
			AppointmentAddrLine2 \\
			AppointmentAddrLine3	\\
			AppointmentAddrLine4 \\ \hline
			
    \end{longtable}
\end{flushleft}

\begin{flushleft}
    \begin{longtable}{|p{12cm}|}
        \hline
			 \textbf{2NF} \\ \hline
         \textbf{Group} \\ \hline
			PersonID (Primary Key) \\ 
         ClientTitle \\
			ClientFirstName \\
			ClientSurname \\
			ClientAddrLine1 \\
			ClientAddrLine2 \\
			ClientAddrLine3 \\
			ClientAddrLine4 \\
			ClientEmail \\
			ClientPhoneNumber \\
			PlastererTitle \\
			PlastererFirstName \\
			PlastererSurname \\
			PlastererAddrLine1 \\
			PlastererAddrLine2 \\
			PlastererAddrLine3 \\
			PlastererAddrLine4 \\
			PlastererEmail \\
			PlastererPhoneNumber \\
			PlastererDailyRate \\ \hline

			\textbf{Group} \\ \hline
			JobID (Primary Key) \\
			PersonID (Composite Key) \\
         JobDescription \\
			JobAddrLine1 \\
			JobAddrLine2 \\
			JobAddrLine3 \\
			JobAddrLine4 \\
			JobDaysWorked \\
			JobComplete \\ \hline
		
			\textbf{Group} \\ \hline
			JobID (Primary Key) \\ 
			MaterialName \\
			MaterialPrice \\
			JobMaterialsQuantity \\ \hline

			\textbf{Group} \\ \hline
			PersonID (Primary Key) \\
			InvoiceAmountAfterTax \\
			InvoiceAmountPreTax \\
			InvoiceReceived \\
			InvoiceDate \\
			InvoiceText \\
			InvoicePaid \\
			AppointmentDate \\
			AppointmentTime \\
			AppointmentAddrLine1 \\
			AppointmentAddrLine2 \\
			AppointmentAddrLine3	\\
			AppointmentAddrLine4 \\ \hline


    \end{longtable}
\end{flushleft}
\begin{flushleft}
    \begin{longtable}{|p{12cm}|}
                \hline
 			\textbf{3NF} \\ \hline
         \textbf{Group} \\ \hline
			PersonID (Primary Key) \\ 
         ClientTitle \\
			ClientFirstName \\
			ClientSurname \\
			ClientAddrLine1 \\
			ClientAddrLine2 \\
			ClientAddrLine3 \\
			ClientAddrLine4 \\
			ClientEmail \\
			ClientPhoneNumber \\
			PlastererTitle \\
			PlastererFirstName \\
			PlastererSurname \\
			PlastererAddrLine1 \\
			PlastererAddrLine2 \\
			PlastererAddrLine3 \\
			PlastererAddrLine4 \\
			PlastererEmail \\
			PlastererPhoneNumber \\
			PlastererDailyRate \\ \hline

			\textbf{Group} \\ \hline
			JobID (Primary Key) \\
			PersonID (Composite Key) \\
         JobDescription \\
			JobAddrLine1 \\
			JobAddrLine2 \\
			JobAddrLine3 \\
			JobAddrLine4 \\
			JobDaysWorked \\
			JobComplete \\ \hline
		
			\textbf{Group} \\ \hline
			JobID (Primary Key) \\ 
			MaterialID (Foreign Key) \\
			JobMaterialsQuantity \\ \hline


			\textbf{Group} \\ \hline
			MaterialID (Primary Key) \\
			MaterialName \\
			MaterialPrice \\ \hline


			\textbf{Group} \\ \hline
			InvoiceID (Primary Key) \\
			InvoiceAmountAfterTax \\
			InvoiceAmountPreTax \\
			InvoiceReceived \\
			InvoiceDate \\ 
			InvoiceText \\
			InvoicePaid \\	\hline

		\textbf{Group} \\ \hline 
			AppointmentID (Primary Key) \\
			AppointmentDate \\
			AppointmentTime \\
			AppointmentAddrLine1 \\
			AppointmentAddrLine2 \\
			AppointmentAddrLine3	\\
			AppointmentAddrLine4 \\ \hline


			\textbf{Group} \\ \hline
			PersonID (Primary Key) \\
			AppointmentID (Foreign Key) \\
			InvoiceID (Foreign Key) \\ \hline

			
    \end{longtable}
\end{flushleft}




\pagebreak
\subsection{SQL Queries}
\begin{flushleft}
Below are some of the most important SQL Queries that will be used in the proposed system.
\end{flushleft}


\textbf{Getting JobMaterials for Invoice Calculation}
\begin{flushleft}
This query selects all the JobMaterials entries for a specific job so that it can be used in the calculation for the entire cost of the job. This cost is then used in the invoice to tell the client how much is due to be paid.
\end{flushleft}

\begin{sql}
SELECT MaterialID,JobMaterialQuantity FROM JobMaterials WHERE JobID = ? 
\end{sql}


\textbf{Get Clients by Search Term}
\begin{flushleft}
This query selects all the clients from the Clients table to be displayed in the results table when searching for clients by a specific search term in the ClientFirstName,ClientSurname and ClientAddrLine2 fields.
\begin{sql}
SELECT * FROM Client WHERE CONTAINS(ClientFirstName, "%?%") OR 
CONTAINS(ClientSurname,"%?%") OR CONTAINS(ClientAddrLine2, "%?%")    
\end{sql}
\end{flushleft}

\textbf{Creating the Job Table}
\begin{flushleft}
This query is an example of an SQL query that will create the Job table in the proposed systems database. This query uses foreign keys to reference primary keys in other related tables.
\begin{sql}
CREATE TABLE Job(
JobID integer,
ClientID integer,
PlastererID integer,
InvoiceID integer,
JobDescription text,
JobAddrLine1 text,
JobAddrLine2 text,
JobAddrLine3 text,
JobAddrLine4 text,
JobDaysWorked integer,
JobComplete text,
Primary Key(JobID),
Foreign Key (ClientID) references Job(JobID),
Foreign Key (PlastererID) references Plasterer(PlastererID),
Foreign Key (InvoiceID) references Invoice(InvoiceID));
\end{sql}
\end{flushleft}

\textbf{Updating the plasterer table by ID}
\begin{flushleft}
This is an example of an SQL query that will be used to update the plasterers info based on the unique primary key that each plasterer has. The :values will be binded to the SQL statement for improved security from SQL injections.
\begin{sql}
UPDATE Plasterer SET 
PlastererTitle = :plastererTitle,
PlastererFirstName = :plastererFirstName,
PlastererSurname = :plastererSurname,
PlastererAddrLine1 = :plastererStreet,
PlastererAddrLine2 = :plastererTown,
PlastererAddrLine3 = :plastererCounty,
PlastererAddrLine4 = :plastererPostCode,
PlastererEmail = :plastererEmail,
PlastererPhoneNumber = :plastererPhoneNumber,
PlastererDailyRate = :plastererDailyRate
WHERE PlastererID = :plastererID;  
\end{sql}
\end{flushleft}

\section{Security and Integrity of the System and Data}

\subsection{Security and Integrity of Data}

\begin{flushleft}

The proposed system will need to protected to the standards put in place by government legislation. The Data Protection Act states the all data must conform to the following rules:

\begin{itemize}
\item Data must be kept up to date.
\item Data must not be kept longer than required.
\item Data must be used in an adequate and relevant manner.
\item Data must be kept safe and secure.
\end{itemize}

In order to abide to these guidlines I must ensure that the database is encrypted so that the accidental loss/theft of the data will not render it dangerous to those it is relevant to.

The proposed system will allow the user to delete records from all tables so that if the data needs to be removed to conform to the data protection act then it can be.

Referential Integrity, which is the ability to perform the same actions to all related pieces of data in the database, will ensure the data is accurate and when, for example, a record is deleted the table entries containing related data is also deleted. This means that the database will behave as expected and not throw errors to the user when they delete a value of a primary key that is also a foreign key elsewhere etc.

\end{flushleft}
\subsection{System Security}

\begin{flushleft}
The access restrictions that will be put in place consists of a login screen where the user enters a strong password which is stored in the database hashed and salted in order to get access to the system.

These restrictions will be put in place in order to keep the personal client data belonging to living individuals secure. The Data Protection Act legislates that all personal data must be stored securely and by developing an integrated login system only an authenticated end user will be able to gain access.

\end{flushleft}

\section{Validation}

\begin{flushleft}
Below is a table showing the validation techniques used for the data items in the proposed system.
\end{flushleft}

\begin{longtable}{|p{3cm}|p{3cm}|p{3cm}|p{3cm}|}
\hline \textbf{Item} & \textbf{Example} & \textbf{Validation} &\textbf{Justification} \\ \hline
ClientTitle & Mr & Presence Check and Matches Title Formats  & Makes sure that the title exists. \\ \hline
ClientFirstName & Kyle & Presence Check and Length Check & Makes sure that the first name is of sufficient length and exists. \\ \hline
ClientSurname & Kirkby & Presence Check and Length Check & Makes sure that the surname is of sufficient length and exists. \\ \hline
ClientAddrLine1 & 15 The Glebe & Presence Check & Makes sure that the street entered. \\ \hline
ClientAddrLine2 & Haverhill & Presence Check and Length Check & Makes sure that the town/city is of a sufficient length and exists. \\ \hline
ClientAddrLine3 & Suffolk & Presence Check and Length Check & Makes sure that the county is of sufficient length and exists. \\ \hline
ClientAddrLine4 & CB9 0DL & Presence Check and PostCode RegEx & Makes sure that the postcode is correctly formatted and that it exists. \\ \hline
ClientEmail & kylekirkby @googlemail.com & Presence Check and RegEx Email Format & Makes sure that the email is in the correct format. \\ \hline
ClientPhoneNumber & 07809726811 & Length Check of 11 digits & Checks that the phone number is 11 digits long. \\ \hline \hline


PlastererTitle & Mr & Presence Check and Matches Title Formats  & Makes sure that the title exists. \\ \hline
PlastererFirstName & Daniel & Presence Check and Length Check & Makes sure that the first name is of sufficient length and exists. \\ \hline
PlastererSurname & Austin & Presence Check and Length Check & Makes sure that the surname is of sufficient length and exists. \\ \hline
PlastererAddrLine1 & 17 Manor Grove & Presence Check & Makes sure that the street entered. \\ \hline
PlastererAddrLine2 & Thurlow & Presence Check and Length Check & Makes sure that the town/city is of a sufficient length and exists. \\ \hline
PlastererAddrLine3 & Suffolk & Presence Check and Length Check & Makes sure that the county is of sufficient length and exists. \\ \hline
PlastererAddrLine4 & CB7 0JL & Presence Check and PostCode RegEx & Makes sure that the postcode is correctly formatted and that it exists. \\ \hline
PlastererEmail & danielaustin133 @gmail.com & Presence Check and RegEx Email Format & Makes sure that the email is in the correct format. \\ \hline
Plasterer PhoneNumber & 07809726811 & Length Check of 11 digits & Checks that the phone number is 11 digits long. \\ \hline
PlastererDailyRate & 200.00 & Check it is a float and Presence Check & Checks that the phone number is 11 digits long. \\ \hline \hline

JobAddrLine1 & 17 Manor Grove & Presence Check & Makes sure that the street entered. \\ \hline
JobAddrLine2 & Thurlow & Presence Check and Length Check & Makes sure that the town/city is of a sufficient length and exists. \\ \hline
JobAddrLine3 & Suffolk & Presence Check and Length Check & Makes sure that the county is of sufficient length and exists. \\ \hline
JobAddrLine4 & CB7 0JL & Presence Check and PostCode RegEx & Makes sure that the postcode is correctly formatted and that it exists. \\ \hline
JobDaysWorked & 5 & Check it is an integer & Makes sure the number of days worked is a whole number. \\ \hline
JobComplete & False & Boolean Data Type & Makes sure it is either True Or False \\ \hline \hline


MaterialName & Plasterboard & Presence Check and Length Check & Makes sure it exists and that it is more than 4 characters etc. \\ \hline
MaterialPrice & 12.00 & Presence Check and Check it is a float & Makes sure it is present before inserted to database and make sure it is a float data type. \\ \hline

JobMaterialsQuantity & 4 & Presence Check and Data Type Check & Makes sure it exists and is an integer \\ \hline \hline

InvoiceAmountPreTax & 100.00 & Presence Check and Float Data Type Check & Makes sure it exists and that it is a float \\ \hline 
InvoiceAmountAfterTax & 120.00 & Presence Check and Float Data Type Check & Makes sure it exists and that it is a float \\ \hline 
InvoiceDate & 14th June 2015 & Presence Check and Date Format Check & Ensures the date is in the correct format and that it exists \\ \hline \hline

AppointmentDate & 19th August 2016 & Presence Check and Date Format Check & Ensures the date is in the correct format and that it exists \\ \hline 
AppointmentTime & 15:20 & Presence Check and Time Format Check & Ensures the time is in the correct format and that it exists \\ \hline 


\end{longtable}

\section{Testing}

\begin{landscape}
\subsection{Outline Plan}

\begin{center}
\begin{tabular}{|p{2cm}|p{5cm}|p{5cm}|p{4cm}|}
\hline
\textbf{Test Series} & \textbf{Purpose of Test Series} & \textbf{Testing Strategy} & \textbf{Strategy Rationale}\\ \hline
1 & User Interface Flow & Top Down Testing & Top Down testing has been chosen due to the structure of the user interface and a top down approach would suit this the best. \\ \hline
2 & User Input Validation & Bottom Up Testing & This test series needs a bottom up approach as the data must be entered and the validation of this data tested so that the system will function accordingly. \\ \hline
3 & Test Data Input into Database & White box testing & White Box Testing has been chosen as I will need a backend view of the database that the user will not be able to do. This needs to be done to verify that the data being inputted is stored correctly in the database. \\ \hline
4 & Test Data Output Functions & Black box testing & Black Box Testing has been chosen as I will need to test the functions that output data in the proposed system from an end users perspective without having a "behind the scenes" view. \\ \hline
5 & Specification Tests & Acceptance Testing & This acceptance test is needed to ensure that the end product meets the initial proposed specifications. \\ \hline
\end{tabular}
\end{center}

\subsection{Detailed Plan}

\begin{center}
    \begin{longtable}{|p{1.5cm}|p{2.5cm}|p{2.5cm}|p{2cm}|p{2cm}|p{2cm}|p{2cm}|p{2cm}|}
        \hline
        \textbf{Test Series} & \textbf{Purpose of Test} & \textbf{Test Description} & \textbf{Test Data} & \textbf{Test Data Type (Normal/ Erroneous/ Boundary)} & \textbf{Expected Result} & \textbf{Actual Result} & \textbf{Evidence}\\ \hline


        1.01 & Test the flow of control for the client ui. & The page buttons should navigate to their respective linked pages. & Click the Add Client and Manage Clients Buttons & Normal & Clicking the buttons should change the ui to a different layout to add or manage clients. & & \\ \hline
		1.02 & Test the flow of control for the plasterers ui. & The buttons on the plasterers menu should change the ui to different layouts to add or manage the plasterers. & Click the Add Plasterer and Manage Plasterers Buttons & Normal & Pressing the Buttons should change the ui of the application so the user can add or manage the plasterers. & & \\ \hline
		1.03 & Test the flow of control for the jobs ui. & The Buttons on the Jobs Menu should change the ui so the end user can add a job and manage jobs & Click the New Job and Manage Jobs Buttons. & Normal & UI should change so the user can add or manage jobs & & \\ \hline

		1.04 & Test the flow of control for loading or creating a database & The Buttons when the application is loaded that will allow the user to Open or Create a Database should change the ui accordingly. & Click the New and Open Database buttons & Normal & Clicking New Database should shows the user a file dialog and then load the main program and open database button should open an existing db and then show the main program ui. & & \\ \hline \hline



	2.01 & Adding a client data validation & Testing the add client form to see if the data inputs are validated correctly. &  Enter usual data for all fields in the adding client form. & Normal & All fields should be highlighted green to show data entry is valid & & \\ \hline
	2.02 & Adding a client erroneous data validation & Testing the add client form by entering erroneous data into form fields & Entering data that should not be accepted in the field. Text in an integer only field for example. & Erroneous & Fields should be highlighted red to show that the data entered is not valid & & \\ \hline
	2.03 & Adding a client boundary data validation & Testing the add client form fields with boundary data input & Entering data into the form fields that is borderline accpetable in the field. & Boundary & These values depending on whether they are in or out the boundaries should highlight the field green if valid and red if not & & \\ \hline
	2.04 & Editing a client data validation & Testing the edit client ui form by entering normal data & Normal Different data should be entered into the edit client form fields & Normal & Form fields should highlight green to show that the data that was entered was valid & & \\ \hline
	2.05 & Editing a client erroneous data validation & Testing the edit client ui form by entering erroneous data & Erroneous values should be typed entered into the edit client form fields & Erroneous & The form fields should highlight red to show that the data entered has not been accepted as valid & & \\ \hline
	2.06 & Editing a client boundary data validation & Testing the edit client ui form by entering boundary data & Boundary data (borderline acceptable values) should be entered into the edit client form fields & Boundary & The forms should highlight according to whether or not the data is in or outside the boundary set for that particular form field & & \\ \hline
	2.07 & Adding a plasterer data validation & Testing the add plasterer form with normal data & Normal data should be entered into the add plasterer form & Normal & The add plasterer form fields should be highlighted green to show the data entered was valid & & \\ \hline
	2.08 & Adding a plasterer erroneous data validation & Testing the add plasterer form with erroneous data & Erroneous data should be entered into each of the form fields on the add plasterer ui. & Erroneous & The form fields should highlight red to show that the data has not been accepted as valid & & \\ \hline
	2.09 & Adding a plasterer boundary data validation & Testing the add plasterer form with boundary data & Boundary values should be entered into the add plasterer form fields & Boundary & The form fields should highlight green or red depending on whether the data is in or outside the boundary set for that field & & \\ \hline
	2.10 & Editing a plasterer data validation & Testing the Edit Plasterer UI with Normal Data & Normal data will be entered into the edit plasterer form fields & Normal & The form fields should highlight green to show the new data is valid & & \\ \hline
	2.11 & Editing a plasterer erroneous data validation & Testing the Edit Plasterer Form with Erroneous Data & Erroneous values should be entered into the form fields & Erroneous & The form fields should highlight red to show the unacceptable data is not valid. & & \\ \hline
	2.12 & Editing a plasterer boundary data validation & Testing the Edit Plasterer Form with Boundary Data & Boundary values should be entered into the form fields & Boundary & The form fields should highlight green or red depending on whether they were in or out the boundary set for that field. & & \\ \hline
	2.13 & Creating a New Job Data Validation & Testing the New Job form with Normal Data & Entering Normal New Job data into the new job form fields & Normal & The New Job form fields should be highlighted greent to show the data enterd is valid. & & \\ \hline
	2.14 & Creating a New Job Erroneous Data Validation & Testing the New Job form with Erroneous Data & Enter erroneous values into the new job form fields & Erroneous & The new job form fields should be highlighted red to show the data entered is invalid. & & \\ \hline
	2.15 & Creating a New Job Boundary Data Validation & Testing the New Job form with Boundary Data & Enter boundary values into the new job form & Boundary & The new job form fields should be highlighted green if the boundary data entered was inside the boundary set for that field. & & \\ \hline \hline


3.01 & Adding New Client Data to Database & Testing whether or not the data is entered into the database correctly. & Valid New Client Data should be entered into the new client form & Normal & The Data should be in the correct columns in the Client table in the database & & \\ \hline
3.02 & Editing a New Client Data Entry to Database & Testing the Edit Client form to see if the data is added to the database correctly & Modified data should be entered into the Edit client form & Normal & The modified client info should be in the correct fields in the Client Table of the database & & \\ \hline
3.03 & Adding New Plasterer Data to Database & Testing the data entry of adding a new plasterer to the database & New plasterer data should be entered into the new plasterer form & Normal & The new plasterer data should be in the correct fields of the plasterer table. & & \\ \hline
3.04 & Editing Plasterer Data Entry to Database & Testing the edit plasterer form data entry to the database to make sure data is stored correctly & Plasterer data should be entered into to the edit plasterer form & Normal & All the data entered should be in the correct position in the plasterer table of the database & & \\ \hline
3.05 & Add a New Job Data Entry to Database & Testing the new job form data entry to the database to ensure data is stored correctly & New job data should be entered into the new job form & Normal & The database should store the new job data correctly in the job table. & & \\ \hline
3.06 & Editing a Jobs Data Entry to Database & Testing the edit job form to see if the data is entered into the database correctly & The Edit form should be completed with valid data. & Normal & The database should stored the modified job data correctly & & \\ \hline  \hline

4.01 & Testing Printing Invoices & An Invoice will be printed to test its functionality works as expected & Click the Print Invoice button on the Manage Jobs UI for a specific job & Normal & The print preview should be displayed and the invoice should be printed. & & \\ \hline
4.02 & Testing Emailing Invoices & An Invoice will be emailed to test this feature works correctly. & Click the Email Invoice button on the manage jobs ui for a specific job & Normal & Check the recipient email inbox for a invoice email & & \\ \hline
4.03 & Test Pay Graph & Testing the pay graph feature of the application & Click the show pay graph button on a specific manage plasterer ui. & Normal & A graph should be displayed for the selected plasterer & & \\ \hline \hline


5.01 & Check Specification is met & Checking the application against the proposed specifications & Going through the application comparing the proposed system against the implemented system & Normal & System should meet specified requirements & & \\ \hline 
    \end{longtable}
\end{center}
\end{landscape}
