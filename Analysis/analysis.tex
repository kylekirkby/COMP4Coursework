\chapter{Analysis}

\section{Introduction}

\subsection{Client Identification}

	\begin{flushleft}

		My client is 30 year old plasterer Dan Austin who runs his own plastering business known as DnA Plastering. Dan mainly uses his Toshiba laptop (Dual Core Intel with 6 GB Ram and running Windows 8 64 bit) to do basic tasks such as social networking and receiving/sending emails. \par

The current system is a paper based method where he records the prices and measurements of the plastering/screening/rendering jobs he undertakes. Dan works in an around the Suffolk/Essex area but occasionally takes on larger jobs further afield in places such as London or Epping. All the recording and calculations are done by Dan himself and does not require additional assistance in completing these tasks but is looking for a digital solution to the organisation problems faced with the current manual paper method. \par

Dan is looking to introduce a computer based system to replace the current one in order to make keeping track of jobs and pricing up new jobs easier and more efficient. Alongside this he would like to be able to keep information on all of his customers so he can simply search for clients' details and contact information all in one location. He will also be able to look up the jobs that he has done for them to make sending invoices easier and manageable.

		
	\end{flushleft}

\subsection{Define the current system}

	\begin{flushleft}

	The current system in place is a paper/notebook based system where details of clients are stored along with prices of jobs and cost of materials needed etc. The details of the clients include their address, phone number, email, first name and surname. The infromation about the job usually includes the measurements of what needs to be plastered along with how long it will take to complete and if he is taking any labourers to too. Calculations are often also made to work out how much to charge depending on the price he is charging per square meter. This rate often changes depending on the current economy. \par

Once all the calculations are made, he works out how much the materials are going to cost and also how long it will take him to complete the job. Once all these calculations and prices have been evaluated he notifies the client of the price; when the price is confirmed the job is undertaken.	 \par

Finally, Dan writes out an invoice using a standard invoice book purchased from a stationary store to inform the client of the costs and charges of the job. The current folder containing the invoices for his clients is not organised and offers another problem whereby finding information for jobs is difficult due to the inability to search quickly for any given customer.
		
	




	\end{flushleft}

\subsection{Describe the problems}


	\begin{flushleft}
	
	Problems are plentiful in the current system. One of the main problems is keeping valuable client data from being lost or damaged as there is only one hard copy made in a notebook. Another problem with the notebook is not being able to easily search through the details of all the clients to find specific phone numbers or contact details. Using a computer based system would allow Dan to search through his clients efficiently and allow him to make backups of the valuable client and job data. \par 
		



	
	\end{flushleft}

\subsection{Section appendix}

\section{Investigation}

\subsection{The current system}

\subsubsection{Data sources and destinations}

\subsubsection{Algorithms}

\subsubsection{Data flow diagram}

\subsubsection{Input Forms, Output Forms, Report Formats}

\subsection{The proposed system}

\subsubsection{Data sources and destinations}

\subsubsection{Data flow diagram}

\subsubsection{Data dictionary}

\subsubsection{Volumetrics}

\section{Objectives}

\subsection{General Objectives}

\subsection{Specific Objectives}

\subsection{Core Objectives}

\subsection{Other Objectives}

\section{ER Diagrams and Descriptions}

\subsection{ER Diagram}

\subsection{Entity Descriptions}

\section{Object Analysis}

\subsection{Object Listing}

\subsection{Relationship diagrams}

\subsection{Class definitions}

\section{Other Abstractions and Graphs}

\section{Constraints}

\subsection{Hardware}

\subsection{Software}

\subsection{Time}

\subsection{User Knowledge}

\subsection{Access restrictions}

\section{Limitations}

\subsection{Areas which will not be included in computerisation}

\subsection{Areas considered for future computerisation}

\section{Solutions}

\subsection{Alternative solutions}

\subsection{Justification of chosen solution}
